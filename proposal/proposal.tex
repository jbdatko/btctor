%File: focus_paper.tex
\documentclass[letterpaper]{article}


% Required Packages
\usepackage{aaai}
\usepackage{times}
\usepackage{helvet}
\usepackage{courier}
\usepackage{graphicx}
\usepackage{mathtools}
\usepackage{url}

\newcommand{\ah}{{\sc{Anchorhead }}}
\newcommand{\dq}{{\sc Anchorhead}$_{DQ}$ }

\begin{document}

%%%%%%%%%%
% Pdfinfo for PDFTEX
% Uncomment and complete the following for metadata if % your paper is typeset using PDFTEX
\pdfinfo{
/Title (Input Your Paper Title Here)
/Author (John Doe, Jane Doe)
/Subject (Input the Proceedings Title Here)
/Keywords (Input your paper’s keywords here)
}
%%%%%%%%%%
% Section Numbers
% Uncomment if you want to use section numbers
% and change the 0 to a 1 or 2
\setcounter{secnumdepth}{1}
%%%%%%%%%%
% Title, Author, and Address Information
\title{Tor Incentives with Bitcoin}
\author{Josh Datko \and Joseph Heenan\\ Drexel University\\
CS645: Network security\\
}
%%%%%%%%%%
% Body of Paper Begins \begin{document}
\maketitle



\section{Introduction}\label{sec:intro}

Tor is an overlay network that uses onion routing to anonymize
Internet Protocol (IP) communication to defend against traffic
analysis.


\section{Motivation}\label{sec:motivation}

The Tor network is at serious risk of exceeding available capacity,
both in terms of available bridge IP addresses as well as in terms of
exit relay bandwidth\cite{tor-internet-days}. We provide a brief summary of two recent
“bandwidth-based” incentivization schemes designed to help prevent
resource exhaustion of the Tor network: LIRA and BRAIDS. We propose an
alternate, monetary incentivization scheme we’re calling BITTOR and
discuss its pros and cons relative to the approaches of the LIRA and
BRAIDS papers.

This paper is unique insofar as it focuses on a scheme to increase the
numbers of both bridges and relays, and looks at expanding on ideas
from LIRA and BRAIDS and using them for a scheme that offers monetary,
as opposed to bandwidth-based, incentives.

\section{Approach}\label{sec:approach}
\section{Related Work}\label{sec:related}
LIRA and BRAIDS focus on incentivizing relays (users operating servers
exiting the Tor network to the Internet) via higher
performance. BRAIDS for example “allows relays to achieve

75\% lower latency than non-relays for interactive traffic, and 90\%
higher bandwidth utilization for non-interactive
traffic\cite{Jansen:2010:RNT:1866307.1866344}". We however have
concerns that any bandwidth-based incentivization scheme will
motivation enough users to participate in the Tor network to
significantly reduce the effectiveness of traffic correlation
analysis. The goal of this proposal is to outline a monetary
incentivization scheme for Tor, based on Bitcoin, which we believe not
only preserves existing network anonymity but offers the potential for
significantly improved anonymity via increased recruitment of relays
and bridges. We believe that this increased recruitment could decrease
the viability of traffic correlation and manipulation attacks by
several orders of magnitude beyond the risks that exist in the current
network (consisting of low-thousands of both bridges and relays).


Traffic Correlation and Manipulation Attacks in Tor Today
As noted in Feigenbaum et. al, a variety of traffic correlation and
manipulation attacks exist again the current Tor network, for example
an attack in which an adversary “drops packets in a denial-of-service
(DoS) attack aimed at forcing users to move to circuits that the
adversary can deanonymize”.[3]

Challenges with Bandwidth-Based Incentivization Schemes
They may not provide sufficient incentive for wide-scale recruiting of
bridges and relays.
\subsection{Novelty}\label{sec:novelty}

Expand Proposed Research Scheme

Proposal - a new version of Tor (is it possible to make backwards
compatible?) which requires continual Bitcoin donations of clients to
continue participating on the network. These donations are small
enough such that they can be mined on commodity hardware/mobile
phones; the incentive to relays is provided due to the fact that there
are potentially hundreds of thousands of machines mining for
Bitcoin. Users who don’t wish to mine Bitcoin can configure their
client to use pre-purchased Bitcoin.

Benefits:
The “paying” users aren’t a subset of the network that can be easily identified.
Potential Negative Consequences to Incentivizing BITTOR Users:
Potential legal obligation of exit operators to maintain server logs:
\url{http://archives.seul.org/or/talk/Dec-2008/msg00061.html}

\section{Evaluation}\label{sec:evaluation}
\section{Milestones}\label{sec:milestones}






Reference to the phenomenon that paying some users for detracts from
the willingness of other users to do it freely:
http://www.mail-archive.com/tor-talk@lists.torproject.org/msg09754.html



%%%%%%%%%%
% References and End of Paper
\bibliography{bitcoin,tor}
\bibliographystyle{aaai}


\end{document}
%%% Local Variables
%%% mode: latex
%%% TeX-master: t
%%% End: