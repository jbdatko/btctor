%File: focus_paper.tex
\documentclass[letterpaper]{article}


% Required Packages
\usepackage{aaai}
\usepackage{times}
\usepackage{helvet}
\usepackage{courier}
\usepackage{graphicx}
\usepackage{url}


\begin{document}

%%%%%%%%%%
% Pdfinfo for PDFTEX
% Uncomment and complete the following for metadata if % your paper is
% typeset using PDFTEX
\pdfinfo{
/Title (BitTor: Bitcoin incentivized Tor)
/Author (Josh Datko, Joseph Heenan)
/Subject (Tor Incentives with Bitcoin)
/Keywords (tor bitcoin)
}
%%%%%%%%%%
% Section Numbers
% Uncomment if you want to use section numbers
% and change the 0 to a 1 or 2
\setcounter{secnumdepth}{1}
%%%%%%%%%%
% Title, Author, and Address Information
\title{Tor Incentives with Bitcoin}
\author{Josh Datko \and Joseph Heenan\\ Drexel University\\
CS645: Network security\\
}
%%%%%%%%%%
% Body of Paper Begins \begin{document}
\maketitle

\section*{Motivation}\label{sec:motivation}
Tor is an overlay network that uses onion routing to anonymize
Internet Protocol (IP) communication to defend against traffic
analysis.  There are several categories of users that benefit from
this service.  Tor protects normal users from collections attempts for
advertisements or market research and also prevents their Internet
service providers from being aware of their browsing history.
Governments also use Tor for intelligence gathering and to mask their
IP address while conducting sting operations.  Also, Tor is used as a
circumvention tool\cite{tor-internet-days}.  In countries like China
with strong, state-sponsored censorship\cite{china-censorship}, Tor
provides a means to allow access to the unfiltered Internet.

The Tor network is mainly supported by volunteers who donate
computational resources and Internet bandwidth.  Unfortunately, the
Tor network is at risk of exceeding available capacity, both in terms
of available bridges (unlisted relays) as well as in terms of exit
relay bandwidth\cite{tor-internet-days}.  Previous efforts to increase
the number of relays have including public awareness
campaigns\cite{eff-campaign} with some success.  However, Tor still
suffers from a "tragedy of the commons\cite{Hardin_1968_Commons}"
since there are not many incentives to run a relay besides altruism.
Even though users receive better anonymity against some attacks while
running a relay\cite{better-anon-tor-relay}, Tor is still in demand of
more relay operators.

Finding the correct incentive for Tor relays without reducing the
anonymity in Tor is challenging.  There are several proposed incentive
methods however most are focused on improving the Quality of Service
(QoS) for relay operators.  These methods require subtle modifications
to the Tor network to compensate for the adjusted QoS.  We propose a
system to simply increase the number of relay operators by providing
an incentive with monetary payments from Bitcoin.

Bitcoin is a peer-to-peer electronic cash currency that uses a
proof-of-work based model\cite{nakamoto2009bitcoin}.  It has seen
increasing popularity since its release in 2009 and there exists
several exchanges to convert Bitcoin in various national currencies.
New Bitcoins enter the market by "mining."  However, as there are a
fixed number of Bitcoins, successive Bitcoins become harder (take more
computational resources) to mine.  To account for the increased
difficulty, computers join a mining
pool\cite{DBLP:journals/corr/abs-1112-4980} to contribute to larger
computation resource, however the reward is now shared amount all
miners.


\section*{Approach}\label{sec:approach}

The goal of this research is to assess the feasibility of a Bitcoin
Tor incentive system, which we are calling BitTor.  First, we intend
to identify the problems and challenges with such a design.  In
addition to the anonymity analysis, we indent to comment on other
issues like possible data retention problems introduced through
payments\cite{data-retention}.  Unfortunately, monetary incentives may
actually reduce the number of relays since volunteers may be
demotivated to provide the service for
free\cite{RePEc:eee:joepsy:v:30:y:2009:i:3:p:500-508} and we plan to
provide on discussion on this issue as well.


We will consider two Bitcoin payments
methods.  One using Zerocoin\cite{Miers:2013:ZAD:2497621.2498124}, an
anonymous Bitcoin approach.  The other will outline an approach with
the pseudo-anonymous Bitcoin and evaluate if it hinders the anonymity
of Tor.  Bitcoin is not anonymous\cite{journals/iacr/AndroulakiKRSC12}
and should not be combined with Tor in a manner that exposes Tor users.

A potential approach for Bitcoin payment is to have the Tor client
also join a decentralized, peer-to-peer mining pool such as P2Pool.
Ideally, as a results of the client's work, the mining pool would pay
the bridge or relay directly without exposing the client.  We call
this type of payment a ``charity payment'' since a Bitcoin account (in
this case the Tor Server) receives payment without performing the
mining.  A more complicate scenario is when a client has Bitcoin and
wants to spend it without mining, since it is not obvious how to pay
the Tor server which revealing the Tor user.  A priority for the
research is to outline a system for this use-case without compromising
anonymity.

As a result of payment scheme investigation, we intend analyze the
trade-offs of required Bitcoin servers or ones that optionally accept
payments.  Servers that require payment may help categorize Tor users
(and thus degrade anonymity).  Perhaps a payment system that is
randomized can improve anonymity will still offering a certain
expected value of benefit.

Then we will compare our incentive system with other systems such as
BRAIDS\cite{Jansen:2010:RNT:1866307.1866344},
PAR\cite{raykova-pet2008}, and LIRA\cite{lira-ndss13}.  Following a
comparison, we will provide our solutions to the identified problems
and analyze the impact on anonymity.  The anonymity goal is to not
degrade the protection provided by Tor by introducing the Bitcoin
incentive.

Lastly, we will prototype the system and analyze the results.  The
system will required a modified Tor client that potentially includes
Bitcoin mining pool software and a modified relay that accepts the
Bitcoins.  Incentives have been shown to increase Tor performance
\cite{"Johnny"Ngan:2010:BIT:2163571.2163590}, so we intend to focus on
the performance of our Bitcoin payment protocol in Tor.  As
manipulative experiments should not be run on the live Tor network, a
standalone network will be used.


\section*{Related Work}\label{sec:related}
Tor projects leaders have identified the need for additional relays
and have suggested that incentives should be
considered\cite{"Johnny"Ngan:2010:BIT:2163571.2163590}.  In that same
paper they also present the ``Gold Star'' incentive system, which
unfortunately creates a noticeable subset of prioritized Tor users.  A
payment incentive system was suggested by PAR\cite{raykova-pet2008},
however it relied on centralized bank which compromises anonymity.  A
QoS incentive scheme, BRAIDS, was proposed that claimed it “allows
relays to achieve 75\% lower latency than non-relays for interactive
traffic, and 90\% higher bandwidth utilization for non-interactive
traffic\cite{Jansen:2010:RNT:1866307.1866344}".  A related QoS
incentive system was introduced with LIRA that was based on a
cryptographic lottery\cite{lira-ndss13}.

Bitcoin privacy has been evaluated and it has been shown that
approximately 40\% of users can be
de-anonymized\cite{journals/iacr/AndroulakiKRSC12}.  Using networking
flow analysis combined with the inherent public transaction ledger, it
is also possible to identify Bitcoin users\cite{journals/corr/abs-1107-4524}.


\subsection{Novelty}\label{sec:novelty}
This is the first research to consider Bitcoin as the incentive
mechanism for Tor.  Unlike LIRA and BRAIDS, it does not intended to
redefine the QoS and unlike PAR, it does not use a centralized bank.

We believe that a monetary incentive will increase relay and bridge
participation more so than QoS based incentives.  While increased QoS
is attractive for certain users, we feel that monetary incentives
have wider appeal and will scale with increased usage in Tor.

If feasible, the mining pool payment to a node that did not perform
the proof-of-work is novel.  Normally, the Bitcoin miner desires
payment for its efforts in computation, but we are not aware of a
charity-like payment mechanism in Bitcoin.


\section*{Evaluation}\label{sec:evaluation}

The goal of research is propose a Bitcoin incentivized Tor system and
since we are introducing a new aspect into Tor, it is expected that we
provide a thorough security analysis of the change in anonymity.
While we believe that Bitcoin can provide an incentive to running
relays, it may turn out that the pseudo-anonymity in Bitcoin is too
degrading for use in Tor.  In that case, we think the study of its
faults is still useful as no such research currently exists.

As this research involves two live systems, Bitcoin which directly
involves finicial transactions and Tor which at minimum, contains data
that users consider private, care must be taken when prototyping.
Demonstrating that a Bitcoin miner can pay another party via a charity
payment is sufficient to demonstrate the Tor use case.  A Tor circuit,
preferably on an isolated network can then be modified to show the
proof of concept for the BitTor prototype.


\section*{Milestones}\label{sec:milestones}

Table \ref{tab:milestones} contains a proposed timeline for completing
this project.  This work project could be easily split among two
people where one person focuses on the Bitcoin mining and the other
the modifications into Tor.  The greatest risk to this feasability
investigation is Bitcoin charity mining.  Given that Bitcoin is
subject to traffic analysis, lacking a direct mining-pool-to-relay
system, it is significantly harder to preserve anonymity.


\begin{table}
  \centering
  \begin{tabular}{l | c}
    Milestone & Week Number\\
    \hline

    Identify Bitcoin Payment Method & 2\\
    Investigate charity Bitcoin mining & 3\\
    Compare method to other incentive methods & 4 \\
    Conduct Anonymity analysis & 5 \\
    Implement Bitcoin charity miner & 7 \\
    Implement Tor Bitcoin extensions & 8 \\
    Implement Prototype & 9 \\
    Analyze Results & 10

  \end{tabular}
  \caption{Project Milestones}
  \label{tab:milestones}
\end{table}





%%%%%%%%%%
% References and End of Paper
\bibliography{bitcoin,tor}
\bibliographystyle{aaai}


\end{document}
%%% Local Variables
%%% mode: latex
%%% TeX-master: t
%%% End:
